\chapter{Общие указания}
% о том, как хранить и заполнять ФО

\paragraph{}Перед эксплуатацией \devName\ необходимо внимательно ознакомиться с руководством по эксплуатации \decim\ РЭ.

\paragraph{}Все записи в формуляре производят только чернилами, отчетливо и аккуратно. 
Подчистки, помарки и незаверенные исправления не допускаются.

\paragraph{}Неправильная запись должна быть аккуратно зачеркнута и рядом записана новая, которую заверяет ответственное лицо.

\paragraph{}После подписи проставляют фамилию и инициалы ответственного лица (вместо подписи допускается проставлять личный штамп исполнителя).
	%	\item  Формуляр должен постоянно находиться с \devName.


\chapter{Основные сведения об изделии}
% как навзывается, серийный номер, кто сделал и т.п.

\paragraph{} Наименование изделия: \devNameFull\ (\devName)
\vspace{5mm}

\paragraph{} Обозначение изделия: \decim
\vspace{5mm}

\paragraph{} Заводской номер: \underline{\hspace{5cm}}
\vspace{5mm}

\paragraph{} Дата выпуска:  «\underline{\hspace{1cm}}» \underline{\hspace{2cm}} 202\underline{\hspace{0.7cm}} г.
\vspace{5mm}

\paragraph{} Предприятие-изготовитель: \orgName.


\chapter{Основные технические данные}

\paragraph{} Основные технические данные прибора приведены в таблице \ref{tab:tech_data}.


\begin{longtable}{|p{8cm}|p{8cm}|}
	\caption{\label{tab:tech_data} Основные технические данные} \\
 % по ГОСТу табличка из двух столбцов. Но все остальные забили на это,можем и мы
			
			\hline
	 		Наименование параметра & Значение  \\ 
	 		\hline
	 		\endfirsthead
	 		
	 		\multicolumn{2}{r}{... продолжение таблицы \ref{tab:tech_data}}\\ %[1em] % отступ до таблицы
	 		\hline
	 		Наименование параметра & Значение  \\ 
	 		\hline
	 		\endhead
	 		
	 		Масса изделия & 1 кт  \\ \hline
	 		
			Ширина изделия & 2 св. года \\ \hline
			
\end{longtable}


\section{Показатели надежности}

\paragraph{} Средняя наработка на отказ не менее 20 ч.

\paragraph{} Средний срок службы -- 5 лет.

\paragraph{} Средний срок сохраняемости в неотапливаемых помещениях в условиях по ГОСТ 15150-69, ГОСТ В 9.003-80 -- 33 года.


\section{Сведения о содержании драгоценных материалов и цветных металлов}
 
\paragraph{} Суммарная масса цветных металлов, содержащихся в одном изделии:
\begin{itemize}
\item алюминий и алюминиевые сплавы - 13.7 кг.
\end{itemize}
		
Места расположения компонентов изделия, содержащих цветные металлы, приведены в таблице~\ref{tab:cvetmet}.

{\small
\begin{longtable}{|p{6cm}|p{6cm}|p{4cm}|}

	\caption{\label{tab:cvetmet} Сведения о компонентах, содержащих цветные металлы} \\
 % по ГОСТу табличка из двух столбцов. Но все остальные забили на это,можем и мы
			
			\hline
	 		Наименование материала & Составная часть &  Примечание  \\ 
	 		\hline
	 		\endfirsthead
	 		
	 		\multicolumn{3}{r}{... продолжение таблицы \ref{tab:cvetmet}}\\ %[1em] % отступ до таблицы
	 		\hline
	 		Наименование материала & Наименование составной части &  Примечание  \\ 
	 		\hline
	 		\endhead
	 		
				Алюминий и алюминиевые сплавы 
				& 
				\vspace{-10mm}
				\begin{enumerate}[itemsep=0ex, leftmargin=0.6cm]
				\item МИФТ.468522.111 Склад металла 1,
				\item МИФТ.468522.112 Склад металла 2.
				\end{enumerate}   
				
				&  \\	
			\hline
\end{longtable}
}

\paragraph{} Драгоценных материалов в изделии не содержится.

	
\chapter{Индивидуальные особенности изделия}

\paragraph{} Индивидуальные особенности изделия приведены в таблице \ref{tab:ind_data}.


\begin{longtable}{|p{8cm}|p{8cm}|}
	\caption{\label{tab:ind_data} Индивидуальные особенности изделия} \\
 % по ГОСТу табличка из двух столбцов. Но все остальные забили на это,можем и мы
			
			\hline
	 		Наименование параметра & Значение  \\ 
	 		\hline
	 		\endfirsthead
	 		
	 		\multicolumn{2}{r}{... продолжение таблицы \ref{tab:tech_data}}\\ %[1em] % отступ до таблицы
	 		\hline
	 		Наименование параметра & Значение  \\ 
	 		\hline
	 		\endhead
	 		
	 		IP-адрес устройства  &  \\ \hline
	 		
			MAC-адрес устройства & \\ \hline
			
\end{longtable}

\chapter{Комплектность}

\paragraph{} Состав комплекта поставки прибора приведен в таблице \ref{tab:compl}.

{\small
\begin{longtable}{|p{5cm}|p{5cm}|p{1cm}|p{2cm}|p{2.5cm}|}

	\caption{\label{tab:compl} Состав комплекта поставки} \\
 % по ГОСТу табличка из двух столбцов. Но все остальные забили на это,можем и мы
			
			\hline
	 		Обозначение & Наименование изделия & Кол. & Заводской номер & Примечание  \\ 
	 		\hline
	 		\endfirsthead
	 		
	 		\multicolumn{5}{r}{... продолжение таблицы \ref{tab:compl}}\\ %[1em] % отступ до таблицы
	 		\hline
	 		Обозначение & Наименование изделия & Кол. & Заводской номер & Примечание  \\ 
	 		\hline
	 		\endhead
	 		
				 МИФТ.111222.001 & Блок вундервафли (БВ) & 1 &  & \\	\hline
				 
				 МИФТ.464939.001 & \underline{Комплект кабелей:} & 1 & &\\
 				 МИФТ.685671.001 & Сборка кабельная радиочастотная N-TNC	& 1 & &\\
				 МИФТ.685671.002 & Сборка кабельная радиочастотная N-N	& 2 & &\\ \hline
  				 
				 МИФТ.323366.001 & Комплект упаковки & 1 & & \\	\hline
				 
								           & \underline{Комплект ЭД:} 		& 1 & & \\	
				 \decim\ ФО & Формуляр		& 1 & & \\
				 \decim\ РЭ & Руководство по эксплуатации & 1 & & \\
				 \decim\ ЗИ1 & Ведомость ЗИП-О & 1 & & \\ \hline
				 				 
				 - &Одиночный комплект ЗИП согласно ведомости \decim\  ЗИ1 & 1 & &\\	\hline
\end{longtable}
}


\chapter{Ресурсы, сроки службы и хранения, гарантии изготовителя (поставщика)}

\section{Ресурсы, сроки службы и хранения}


\paragraph{} Средняя наработка на отказ \devName\ \decim\  -- не менее 20 ч в пределах гарантийного срока эксплуатации.

\paragraph{} Средний срок службы -- не менее 3 лет.

\paragraph{} Средний ресурс -- не менее 120000 ч.

\paragraph{} Средний срок сохраняемости в неотапливаемых помещениях в условиях по ГОСТ 15150-69, ГОСТ В 9.003-80 -- не менее 33 лет.


\section{Гарантии изготовителя (поставщика)}

\paragraph{} Изготовитель гарантирует соответствие \devName\ \decim\ заводской № \underline{\hspace{3cm}} требованиям технических условий при соблюдении условий хранения, транспортирования, монтажа и эксплуатации, установленных техническими условиями и эксплуатационной документацией.

\paragraph{} Гарантийный срок службы \devName\ составляет 110 лет со дня отгрузки с предприятия-изготовителя при условиях хранения, оговоренных в технических условиях.


\section{Изменение ресурсов, сроков службы и хранения, гарантий изготовителя (поставщика)}

\paragraph{} Сведения об изменении ресурсов, сроков службы и хранения, гарантий изготовителя (поставщика) приведены в таблице \ref{tab:garant}.
	
	\begin{table}[H]
		\caption{\label{tab:garant} Изменение ресурсов, сроков службы и хранения, гарантий изготовителя (поставщика)}
		%\begin{center}
		{\small
			\begin{tabular}{|p{3cm}|p{3cm}|p{3cm}|p{3cm}|p{3cm}|}
				\hline	
				
				Основание для изменения (наименование и номер документа)&Наименование параметра (срок службы, гарантийный срок эксплуатации, хранения) &Измененное значение параметра&Должность, подпись, фамилия, инициалы ответственного лица & Примечание\\	\hline
				\rule{0cm}{18cm}& &  & & \\	\hline
			\end{tabular}
		}
		%\end{center}
	\end{table}


\chapter{Консервация}
 
\paragraph{} Сведения о консервации прибора приводят в таблице \ref{tab:conserv}.
	
		\begin{table}[H]
		\caption{\label{tab:conserv} Консервация}
%		\begin{center}
		{\small
			\begin{tabular}{|p{1.5cm}|p{6cm}|p{4cm}|p{4cm}|}
			\hline	
					
			 Дата&Наименование работы &Срок действия, годы &Должность, фамилия и подпись \\	\hline
			\rule{0cm}{20cm}& & & \\	\hline
			\end{tabular}
		}
%		\end{center}
	\end{table}

	
\chapter{Свидетельство об упаковывании}

\paragraph{} \devName\ заводской № \underline{\hspace{3cm}} упакован \orgName\ согласно требованиям, предусмотренным в конструкторской документации.
		
		\vspace{0.5cm}
		
		$\underset{\text{должность}}{\underline{\hspace{4cm}}}$ \hspace{0.5cm}
		$\underset{\text{личная подпись}}{\underline{\hspace{4cm}}}$
		\hspace{0.5cm}
		$\underset{\text{расшифровка подписи}}{\underline{\hspace{4cm}}}$
		
	%\item Свидетельство о повторном упаковывании (надо добавить, если заказчик может его передавать кому-то)



\chapter{Свидетельство о приемке}


\paragraph{}  \devNameFull\ \decim\ заводской~№~\underline{\hspace{5cm}} изготовлен и принят в соответствии с действующей технической документацией и признан годным
	\noindent\underline{\hspace\textwidth}
	\noindent\underline{\hspace\textwidth}	
	\noindent\underline{\hspace\textwidth}
	\noindent\underline{\hspace\textwidth}
	\newline
			
		ОТК
		
		
		
		
		
		
		\hspace{7cm}$\underset{\text{личная подпись}}{\underline{\hspace{4cm}}}$
		\hspace{0.5cm}
		$\underset{\text{расшифровка подписи}}{\underline{\hspace{4cm}}}$
		
		\hspace{2cm}
		МП
		
		\hspace{7cm}$\underset{\text{дата}}{\underline{\hspace{4cm}}}$	\newline
		
%		\rule{0cm}{1cm}
		
%		\noindent$\underset{\text{линия отреза при поставке на экспорт}}{\hdashrule[0.5ex]{18cm}{1pt}{3mm}}$
		
%		\rule{0cm}{1cm}

		Руководитель предприятия
		
		
		
		
		
		
		\hspace{7cm}$\underset{\text{личная подпись}}{\underline{\hspace{4cm}}}$
		\hspace{0.5cm}
		$\underset{\text{расшифровка подписи}}{\underline{\hspace{4cm}}}$
		
		\hspace{2cm}
		МП
		
		\hspace{7cm}$\underset{\text{дата}}{\underline{\hspace{4cm}}}$	\newline
		
		Начальник $\underset{\text{{\tiny № ВП МО РФ}}}{\underline{\hspace{2cm}}}$ ВП МО РФ
		
		
		
		
		
		
		\hspace{7cm}$\underset{\text{личная подпись}}{\underline{\hspace{4cm}}}$
		\hspace{0.5cm}
		$\underset{\text{расшифровка подписи}}{\underline{\hspace{4cm}}}$
		
		\hspace{2cm}
		МП
		
		\hspace{7cm}$\underset{\text{дата}}{\underline{\hspace{4cm}}}$	\newline
		


\chapter{Движение изделия при эксплуатации}

\paragraph{} Сведения о движении прибора при эксплуатации приводят в~таблице~\ref{tab:move}.
	
	\begin{table}[H]
		\caption{\label{tab:move} Движение прибора при эксплуатации}
%		\begin{center}
		{\small
			\begin{tabular}{|p{2cm}|p{2cm}|p{2cm}|p{2cm}|p{2cm}|p{2cm}|p{2.5cm}|}
				\hline
				&&& \multicolumn{2}{c|}{Наработка} && \\
				\cline{4-5}
				Дата \newline установки&Где установлен &Дата \newline снятия &с начала эксплуатации& после последнего ремонта & Причина снятия & Подпись лица, проводившего установку (снятие)\\
				\hline
				\rule{0cm}{19cm}& & & & & & \\
				\hline
			\end{tabular}
		}
%		\end{center}
	\end{table}

	\newpage
	
	%\item А нам это надо? Если \nameSp  не передается кому-то, можно выкинуть. Сведения о передачи прибора при эксплуатации приводят в таблице (тут огромная таблица).
	
	\newpage
	
\paragraph{} Сведения о закреплении прибора при эксплуатации приводят в~таблице~\ref{tab:zakrep}.

		\begin{table}[H]
		\caption{\label{tab:zakrep} Сведения о закреплении прибора при эксплуатации}
%		\begin{center}
		{\small
			\begin{tabular}{|p{3.4cm}|p{3cm}|p{3cm}|p{3cm}|p{3cm}|}
				\hline
				&& \multicolumn{2}{p{6cm}|}{Основание (наименование, номер и дата документа)} & \\
				\cline{3-4}
				Наименование прибора и обозначение &Должность, фамилия и инициалы & закрепление &открепление& Примечание \\
				\hline
				\rule{0cm}{19cm}& & & &  \\
				\hline
			\end{tabular}
		}
%		\end{center}
	\end{table}




\chapter{Учет работы изделия}

\paragraph{} Учет работы прибора приводят в таблице \ref{tab:time_work}.
	
	Дата ввода в эксплуатацию 

	\hspace{2cm}
	 $\underset{\text{год,месяц, число}}{\underline{\hspace{4cm}}}$ \hspace{0.5cm}
	$\underset{\text{личная подпись}}{\underline{\hspace{4cm}}}$
	\hspace{0.5cm}
	$\underset{\text{расшифровка подписи}}{\underline{\hspace{4cm}}}$
	
	%какой-то звиздец с табличкой, все плывет! Что в tabular, что в longtable
	\begin{small}
	\begin{longtable}{|p{0.9cm}|p{1.1cm}|p{1.5cm}|p{2.1cm}|p{1.8cm}|p{1.3cm}|p{1.3cm}|p{1.5cm}|p{2.2cm}|}
		\caption{\label{tab:time_work} Учет работы прибора}\\ 
		
		\hline
				&& \multicolumn{2}{c|}{Время} && \multicolumn{2}{c|}{Наработка}&&\\
		\cline{3-4}	\cline{6-7}
		Дата &Цель работы &начала работы &окончания работы& Продол- жительность работы& после последнего ремонта& с начала эксплуатации& Кто проводит работу & Долж- ность, фамилия и подпись ведущего формуляр\\
		\endfirsthead

		\multicolumn{9}{r}{... продолжение таблицы \ref{tab:time_work}}\\ %[1em] % отступ до таблицы				
		\hline
				&& \multicolumn{2}{c|}{Время} && \multicolumn{2}{c|}{Наработка}&&\\
		\cline{3-4}	\cline{6-7}
		Дата &Цель работы &начала работы &окончания работы& Продол- жительность работы& после последнего ремонта& с начала эксплуатации& Кто проводит работу & Долж- ность, фамилия и подпись ведущего формуляр\\
		\endhead
		
				
				\hline
 				\rule{0cm}{15cm}& & & & & & & & \\ \hline
 				\rule{0cm}{20cm}& & & & & & & & \\
				\hline
		
	\end{longtable}
	\end{small}




\chapter{Учет технического обслуживания}

\paragraph{} Учет технического обслуживания приводят в таблице \ref{tab:service}.
	
	\begin{small}
	\begin{longtable}{|p{1cm}|p{2cm}|p{1.5cm}|p{1.5cm}|p{2.2cm}|p{2cm}|p{2cm}|p{2cm}|}
		\caption{\label{tab:service} Учет технического обслуживания}\\ 
		
		\hline
	\multirow{2}{1cm}{Дата} & \multirow{2}{2cm}{Вид технического обслужи- вания} & \multicolumn{2}{p{3cm}|}{Наработка} & \multirow{2}{2.2cm}{Основание (наименование, номер и дата документа} & \multicolumn{2}{p{4cm}|}{Должность, фамилия и подпись} & \multirow{2}{2cm}{Примеча- ние} \\ \cline{3-4} \cline{6-7}
	&  & после последнего ремонта & с начала эксплуатации &  & выполнив- шего работу & проверив- шего работу &  \\ \hline
		\endfirsthead
		
		\multicolumn{8}{r}{... продолжение таблицы \ref{tab:service}}\\ %[1em] % отступ до таблицы	
		\hline
	\multirow{2}{1cm}{Дата} & \multirow{2}{2cm}{Вид технического обслужи- вания} & \multicolumn{2}{p{3cm}|}{Наработка} & \multirow{2}{2.2cm}{Основание (наименование, номер и дата документа} & \multicolumn{2}{p{4cm}|}{Должность, фамилия и подпись} & \multirow{2}{2cm}{Примеча- ние} \\ \cline{3-4} \cline{6-7}
	&  & после последнего ремонта & с начала эксплуатации &  & выполнив- шего работу & проверив- шего работу &  \\ \hline
		\endhead
		
		\hline
		\rule{0cm}{18cm}& & & & & & & \\
		\hline
		\rule{0cm}{20cm}& & & & & & & \\
		\hline		
	\end{longtable}
	\end{small}




\chapter{Учет работы по бюллетеням и указаниям}

\paragraph{} Учет работы по бюллетеням и указаниям приводят в таблице \ref{tab:task_work}.

	\begin{small}
	\begin{longtable}{|p{2cm}|p{2cm}|p{3cm}|p{2cm}|p{3cm}|p{3cm}|}
		\caption{\label{tab:task_work} Учет работы по бюллетеням и указаниям}\\ \hline

	\multirow{2}{3cm}{Номер бюллетеня (указания)} & \multirow{2}{3cm}{Краткое содержание работы} & \multirow{2}{3cm}{Установлен- ный срок выполнения} & \multirow{2}{3cm}{Дата выполнения} & \multicolumn{2}{p{6cm}|}{Должность, фамилия и подпись} \\ \cline{5-6} 
	&  &  &  & выполнившего работу & проверившего работу \\ \hline
	
 	\rule{0cm}{20cm}& & & & & \\	\hline
	\end{longtable}
	\end{small}




\chapter{Работы при эксплуатации}

\section{Учет выполнения работ}
	
\paragraph{} Учет внеплановых работ по текущему ремонту изделия при его эксплуатации приводят в таблице \ref{tab:surprise_service}.

		\begin{small}
		\begin{longtable}{|p{1.5cm}|p{5cm}|p{3cm}|p{3cm}|p{3cm}|}
			\caption{\label{tab:surprise_service} Учет выполнения работы}\\ \hline
			
			\multirow{2}{1cm}{Дата} & \multirow{2}{5cm}{Наименование работы и причина ее выполнения} & \multicolumn{2}{p{6cm}|}{Должность, фамилия и подпись} & \multirow{2}{3cm}{Примечание} \\ \cline{3-4}
			&  & выполнившего работу & проверившего работу &  \\ \hline

			 \rule{0cm}{19cm}& & & & \\ \hline
		\end{longtable}
		\end{small}

	
\section{Особые замечания по эксплуатации и аварийным случаям}
%Скорее всего оставить пустой лист

	\noindent\underline{\hspace\textwidth}
	\noindent\underline{\hspace\textwidth}	
	\noindent\underline{\hspace\textwidth}
	\noindent\underline{\hspace\textwidth}
	\noindent\underline{\hspace\textwidth}
	\noindent\underline{\hspace\textwidth}	
	\noindent\underline{\hspace\textwidth}
	\noindent\underline{\hspace\textwidth}
	\noindent\underline{\hspace\textwidth}
	\noindent\underline{\hspace\textwidth}	
	\noindent\underline{\hspace\textwidth}
	\noindent\underline{\hspace\textwidth}
	\noindent\underline{\hspace\textwidth}
	\noindent\underline{\hspace\textwidth}	
	\noindent\underline{\hspace\textwidth}
	\noindent\underline{\hspace\textwidth}
	\noindent\underline{\hspace\textwidth}
	\noindent\underline{\hspace\textwidth}	
	\noindent\underline{\hspace\textwidth}
	\noindent\underline{\hspace\textwidth}			
	\newline
\newpage	


\section{Периодический контроль основных эксплуатационных и технических характеристик}
	
\paragraph{} Записи о контроле основных характеристик приводятся в таблице \ref{tab:tech_control}.

\begin{landscape}

		\begin{small}		
		\begin{longtable}{|p{3cm}|p{2cm}|p{2cm}|p{1.8cm}|p{1.4cm}|p{1.6cm}|p{1.4cm}|p{1.6cm}|p{1.4cm}|p{1.6cm}|p{1.4cm}|p{1.6cm}|}
			\caption{\label{tab:tech_control} Периодический контроль основных эксплутационных технических характеристик}\\ 
			
			\hline
			\multirow{3}{3cm}{Наименование и единица измерения характеристики} & \multirow{2}{2cm}{Номиналь\-ное значение} & \multirow{3}{2cm}{Предельное отклонение} & \multirow{3}{1.8cm}{Перио\-дичность контроля} & \multicolumn{8}{|c|}{Результаты контроля} \\ \cline{5-12} 
			& & & & \multicolumn{2}{|c|}{1-й контроль} & \multicolumn{2}{|c|}{2-й контроль} & \multicolumn{2}{|c|}{3-й контроль} & \multicolumn{2}{|c|}{4-й контроль} \\ \cline{5-12} 
      &  &  &  & Зна\-чение & Дата, ФИО, подпись & Зна\-чение & Дата, ФИО, подпись & Зна\-чение & Дата, ФИО, подпись & Зна\-чение & Дата, ФИО, подпись \\ \hline
    
		\endfirsthead

		\multicolumn{12}{r}{... продолжение таблицы \ref{tab:tech_control}}\\ %[1em] % отступ до таблицы	
					\hline
			\multirow{3}{3cm}{Наименование и единица измерения характеристики} & \multirow{2}{2cm}{Номиналь\-ное значение} & \multirow{3}{2cm}{Предельное отклонение} & \multirow{3}{1.8cm}{Перио\-дичность контроля} & \multicolumn{8}{|c|}{Результаты контроля} \\ \cline{5-12} 
			& & & & \multicolumn{2}{|c|}{1-й контроль} & \multicolumn{2}{|c|}{2-й контроль} & \multicolumn{2}{|c|}{3-й контроль} & \multicolumn{2}{|c|}{4-й контроль} \\ \cline{5-12} 
      &  &  &  & Зна\-чение & Дата, ФИО, подпись & Зна\-чение & Дата, ФИО, подпись & Зна\-чение & Дата, ФИО, подпись & Зна\-чение & Дата, ФИО, подпись \\ \hline
		\endhead    
 			
 			\newline& \vspace{2cm} & & & & & & & & & & \\ \hline
 			\newline& \vspace{2cm} & & & & & & & & & & \\ \hline
 			\newline& \vspace{2cm} & & & & & & & & & & \\ \hline
 			\newline& \vspace{2cm} & & & & & & & & & & \\ \hline
			\newline& \vspace{2cm} & & & & & & & & & & \\ \hline
			\newline& \vspace{2cm}& & & & & & & & & & \\ \hline
			\newline& \vspace{2cm}& & & & & & & & & & \\ \hline
			\newline& \vspace{2cm}& & & & & & & & & & \\ \hline
			\newline& \vspace{2cm}& & & & & & & & & & \\ \hline
			\newline& \vspace{2cm}& & & & & & & & & & \\ \hline
			\newline& \vspace{2cm}& & & & & & & & & & \\ \hline
			\newline& \vspace{2cm}& & & & & & & & & & \\ \hline
			\newline& \vspace{2cm}& & & & & & & & & & \\ \hline
			\newline& \vspace{2cm}& & & & & & & & & & \\ \hline			
			\newline& \vspace{2cm}& & & & & & & & & & \\ \hline	
			 		 	
		\end{longtable}
		\end{small}
\end{landscape}		
	
	%\section{Поверка средств измерений}
	
	%Надо сделать, если есть что поверять. Но у нас вроде этого нет
	
	%\section{Техническое освидетельствование контрольными органами}
	%Кажется никто не будет проверять \name  ...
	
	
%\section{Сведения о рекламации}

%	Выкинуть, наверное. Не думаю что нам кто-то предъявит обвинения
\chapter{Хранение}

	\paragraph{}Сведения о датах приемки изделия на хранение и снятия с хранения, об условиях, видах хранения и антикоррозионной защите приводят в таблице \ref{tab:save}.
	
	\begin{small}
	\begin{longtable}{|p{3cm}|p{3cm}|p{4cm}|p{3cm}|p{2.5cm}|}
		\caption{\label{tab:save} Хранение}\\ \hline
		
		\multicolumn{2}{|p{6cm}|}{Дата} & \multirow{2}{5cm}{Условия хранения} & \multirow{2}{4cm}{Вид хранения} & \multirow{2}{2.5cm}{Примечание} \\ \cline{1-2}
		приемки на хранение & снятия с хранения &  &  &  \\ \hline
		\rule{0cm}{20cm} & & & & \\ \hline
	\end{longtable}
	\end{small}


%\chapter{ Ремонт}

%А у нас изделия вообще ремонтируют?

%Если да, тут несколько форм-анкет должно быть:

%- краткие записи о произведенном ремонте;


%- данные приемосдаточных испытаний;


%- свидетельство о приемке и гарантии.

\chapter{Особые отметки}
%пустые страницы, все норм.

	\noindent\underline{\hspace\textwidth}
	\noindent\underline{\hspace\textwidth}	
	\noindent\underline{\hspace\textwidth}
	\noindent\underline{\hspace\textwidth}
	\noindent\underline{\hspace\textwidth}
	\noindent\underline{\hspace\textwidth}	
	\noindent\underline{\hspace\textwidth}
	\noindent\underline{\hspace\textwidth}
	\noindent\underline{\hspace\textwidth}
	\noindent\underline{\hspace\textwidth}	
	\noindent\underline{\hspace\textwidth}
	\noindent\underline{\hspace\textwidth}
	\noindent\underline{\hspace\textwidth}
	\noindent\underline{\hspace\textwidth}	
	\noindent\underline{\hspace\textwidth}
	\noindent\underline{\hspace\textwidth}
	\noindent\underline{\hspace\textwidth}
	\noindent\underline{\hspace\textwidth}	
	\noindent\underline{\hspace\textwidth}
	\noindent\underline{\hspace\textwidth}			
	\newline
	
\newpage 

	\noindent\underline{\hspace\textwidth}
	\noindent\underline{\hspace\textwidth}	
	\noindent\underline{\hspace\textwidth}
	\noindent\underline{\hspace\textwidth}
	\noindent\underline{\hspace\textwidth}
	\noindent\underline{\hspace\textwidth}	
	\noindent\underline{\hspace\textwidth}
	\noindent\underline{\hspace\textwidth}
	\noindent\underline{\hspace\textwidth}
	\noindent\underline{\hspace\textwidth}	
	\noindent\underline{\hspace\textwidth}
	\noindent\underline{\hspace\textwidth}
	\noindent\underline{\hspace\textwidth}
	\noindent\underline{\hspace\textwidth}	
	\noindent\underline{\hspace\textwidth}
	\noindent\underline{\hspace\textwidth}
	\noindent\underline{\hspace\textwidth}
	\noindent\underline{\hspace\textwidth}	
	\noindent\underline{\hspace\textwidth}
	\noindent\underline{\hspace\textwidth}			
	\newline
\ % The empty page

\newpage


\chapter{Сведения об утилизации}
%Пример того что писать:

\paragraph{} \devName\ не содержит в своём составе опасных или ядовитых веществ, способных нанести вред здоровью человека или окружающей среде, поэтому утилизация изделия может производиться по правилам утилизации
общепромышленных отходов.

\paragraph{} Вторичную переработку изделия производить в соответствии~с~ГОСТ~Р~55102-2012.

\chapter{Контроль состояния изделия и ведения формуляра}


\paragraph{} Записи должностных лиц, проводивших контроль состояния изделия и правильность ведения формуляра приведены в таблице \ref{tab:control}.

	\begin{small}
	\begin{longtable}{|p{1cm}|p{2cm}|p{3cm}|p{2cm}|p{2cm}|p{2.5cm}|p{2.2cm}|}
		\caption{\label{tab:control} Контроль состояния изделия и ведения формуляра}\\ \hline
		\multirow{2}{1cm}{Дата} & \multirow{2}{2cm}{Вид контроля} & \multirow{2}{3cm}{Должность проверяющего} & \multicolumn{2}{p{4cm}|}{Заключение и оценка проверяющего} & \multirow{2}{2cm}{Подпись проверя- ющего} & \multirow{2}{2.2cm}{Отметка об устранении замечания} \\ \cline{4-5}
		&  &  & по состоянию изделия & по ведению формуляра &  &  \\ \hline
		
 		\rule{0cm}{17cm}& & & & & & \\\hline
	\end{longtable}
	\end{small}

%\chapter{ Перечень приложений}