\chapter{Анализ факторов, ограничивающих точность навигационных определений}
\label{cha:analysis}

Глобальные навигационные системы созданы для предоставления потребителю сервиса: сервиса определения его местоположения и скорости, точной синхронизации и т.д. 
Но качество навигационного сервиса зависит от множества факторов: условий работы, решаемой потребителем задачи и, что особенно важно, используемого потребителем оборудования. 
Не существует <<стандартной>> навигационной аппаратуры потребителей (НАП), приемник строится разработчиками опираясь на интерфейсный контрольный документ (ИКД) и доступные технические средства для реализации обработки навигационных сигналов. 
Результат определяется решаемой НАП задачей (требованиями назначения), структурой навигационного сигнала и уровнем развития электронной техники. 

В данной главе разрабатывается обобщенная модель навигационного сигнала ГНСС, обобщенная модель НАП и выделяются факторы, ограничивающие точность навигационных определений и помехоустойчивость этой НАП.


\section{Техники позиционирования по сигналам ГНСС и задачи с их помощью решаемые}


\subsection{Техники позиционирования: \\SPP, DGNSS, RTK, PPP}

В ходе развития спутниковой навигации появились несколько вариантов реализации псевдодальномерного метода. 
Отличаются они способом измерения задержки сигнала, требованиями к точности информации о положении маяков и тем, как описывается положение потребителя. 
Аналоги этих техник могут использоваться в ЛРНС.

\subsubsection{Абсолютное позиционирование по кодовым измерениям: SPP}

Техника позиционирования \textbf{SPP} (англ. single point positioning) -- это прямая реализация описанного выше псевдодальномерного метода, где в качестве наблюдений псевдодальностей используется задержка огибающей сигнала.

\paragraph{}А ещё у нас есть пункты

\subparagraph{}И подпункты. Да-да, так можно писать контракт, ТЗ или методику.

Почему именно огибающей? 
При формировании сигнала спутник производит преобразование собственной шкалы времени $t^s$ в сигнал $S(t^s)$, используя одно и то же время и для огибающей, и для несущей:
\begin{equation}
S(t^s) = A h(t^{s}) \cos(2\pi f_0 t^s) 
\end{equation}

Но в процессе распространения сигнал проходит среды и устройства, в которых фазовая и групповая скорости немного отличаются.
Примером такой среды выступает ионосфера. 
При этом фаза огибающей и фаза несущей немного расходятся, как результат, наблюдаемое по несущей и по огибающей время отличается на несколько десятков наносекунд. 
Поэтому можно говорить про \textbf{кодовое} $t^{cs}$ (то есть соответствующее огибающей) и \textbf{фазовое} (то есть соответствующее несущей) $t^{ps}$ сигнальное время:
\begin{equation}
S\left(t_r\right) = A h\left(t^{cs}\right) \cos\left(2\pi f_0 t^{ps}\right),\   t^{cs}=t^{cs}\left(t_{r}\right), t^{ps}=t^{ps}\left(t_{r}\right),
%\label{eq:signalTimesModel_fund}
\end{equation}
\noindent где $t_r$ -- время по часам приемника, в которых он и наблюдает принимаемый сигнал. 

Прилагательное "кодовое" (англ. code) используется так как сигнал промоделирован специальным дальномерным кодом, во многом и определяющим функцию $h(t)$. 
Слово "фазовое" несколько неоднозначно, т.к. можно говорить и о фазе кода, и о фазе несущей.
Но по-умолчанию под фазой понимают именно фазу несущего колебания, откуда и пошла практика соответствующие измерения называть фазовыми. 
В англоязычных текстах часто используется более корректное прилагательное ''carrier-phase'', например, carrier-phase measurements \cite[стр. 8, 11]{rinex303_update1}.

Фазовое и кодовое сигнальные времена не только отличаются численно, но и их измерения принципиально отличаются свойствами.
В радионавигации часто проявляется такое свойство измерений, как их \textit{неоднозначность}.
В повседневной жизни мы тоже каждый день встречаемся с этим свойством, когда смотрим на простые часы: так мы можем узнать время дня, но не узнать день недели и текущую дату.
Таким образом, мы получаем точное измерение времени, но с неоднозначностью в сутки. 
Аналогичный эффект возникает при наблюдении за сигналом: 
\begin{itemize}
\item по огибающей можно узнать однозначное время, включая и год, и день, и секунду, и её долю, т.к. огибающая непериодична;
\item наблюдение за несущей дает время с очень маленьким шагом неоднозначности, в долю наносекунды, т.к. у функции $\cos(2 \pi f_0 t)$ маленький период.
\end{itemize}

Использовать кодовые измерения для решения задачи, ввиду их однозначности, значительно проще, чем фазовые. 
Поэтому абсолютное позиционирование по кодовым измерениям реализуется в любом навигационном приемнике, а большинство из них этим подходом и ограничивается. 
Преимущества SPP: простота, надежность и автономность. 
Недостатком является относительная грубость, типичные ошибки позиционирования в этом режиме составляют 5-10 метров. 

Другие техники позиционирования базируются на SPP, но используют дополнительные данные для увеличения точности. 

\subsubsection{Относительное позиционирование по кодовым измерениям: DGNSS}

В технике \textbf{DGNSS} (differencial GNSS) для увеличения точности позиционирования приемник дополнительно использует кодовые измерения от одной или нескольких базовых станций, расположенных на удалении до нескольких десятков километров.
При этом, как правило, данные передаются посредством дополнительного радиоканала или сигналов сотовой сети. 
В качестве \textit{базовых станций} выступают специальные неподвижные ГНСС приемники с точно известными координатами. 

Так как приемник находится недалеко от базовых станций, то он наблюдает тот же набор спутников, а ошибки его измерений и измерений базовой станции во многом совпадают. 
Составляется система уравнений, подобная (\ref{eq:RevPseudoRangeMethod}), но для радиус-вектора, соединяющего базу и приемник (его в этом случае называют \textit{ровером}).
При её решении общие ошибки измерений взаимно компенсируют друг друга, что позволяет определить радиус-вектор с точностью около 1 метра в случае расстояния до базы в несколько километров. 

При удалении от базы корреляция ошибок измерений падает, поэтому погрешность позиционирования возрастает со скоростью около 2-6 см на каждые 10 км удаления \cite{federalRadionavigationPlan2001, monteiro2005}. 

Так как в результате решения системы уравнений получают положение приемника относительно базы, то эту технику относят к классу относительных. 
Но так как координаты базы известны точно, то простым прибавлением измеренного радиус-вектора к ним получают абсолютное положение приемника.

Толчком к развитию DGNSS послужила работа режима селективного доступа (англ. SA, selective availability) на спутниках системы GPS до начала 2000 года. 
В этом режиме в транслируемые спутником эфемериды (информации о положении аппаратов) закладывались дополнительные ошибки, снижающие точность позиционирования до уровня 100 метров. 
В режиме SPP только авторизированные пользователи могли получить высокую точность, т.к. им предоставлялась модель этих искусственных смещений. 
Применение же дифференциального метода по своему принципу позволяло компенсировать ошибки в эфемеридах и получать точные оценки координат всем.

\subsubsection{Относительное позиционирование по фазовым измерениям: RTK и PPK}

Взаимная компенсация целого ряда ошибок измерений в дифференциальном режиме позволяет воспользоваться для определения местоположения не только кодовыми, но и фазовыми измерениями. 
Но для этого всё ещё надо \textit{разрешить неоднозначность} фаз, например, с использованием LAMBDA метода \cite{teunissen1993}.

Фазовые измерения на два порядка точнее кодовых, в результате, если выполняются все условия для получения решения, то его точность составляет около 1 сантиметра.
Как и в случае кодовыми измерениями, удаление ровера от базы приводит к деградации точности, примерно на 1 мм на каждый километр увеличения расстояния. 

В зависимости от времени получения навигационного решения выделяют Static, RTK и PPK режимы. 

В режиме Static ровер оставляют неподвижным на несколько минут, накапливают измерения. 
Сами оценки координат получают с учетом неподвижности ровера и, как правило, уже после проведения сеансов измерений. 
Например, накопленные данные выгружаются в компьютер, имеющий доступ к данным базовой станции через Интернет, который уже и производит расчет положения. 

В RTK и PPK режимах ровер может продолжать перемещения, режимы же отличаются моментом обработки измерений. 


В режиме \textbf{PPK} (Post Processing Kinematic, позиционирование движущегося объекта в режиме пост-обработки) обработка измерений происходит после проведения сеанса измерений, но в этом режиме требования неподвижности ровера не предъявляются. 
Благодаря этому, PPK широко применяется на подвижных объектах, например, для проведения геодезических изысканий с помощью беспилотных летательных аппаратов.
Пост-обработка измерений позволяет исключить прямой канал связи между приемником и базой, что упрощает реализацию. 

В режиме \textbf{RTK} (Real Time Kinematic, позиционирование движущегося объекта в реальном времени) данные от базовой станции передаются роверу по некоторому каналу связи в момент снятия измерений, что позволяет сразу получить точное навигационное решение.
Данный режим широко применяется при проведении наземных геодезических работ людьми, т.к. позволяет выявить возможные проблемы сразу в момент съемки и внести оперативные исправления. 

\subsubsection{Абсолютное позиционирование по фазовым измерениям: PPP}

Группа методов PPP (англ. Precise Point Positioning) приближается к абсолютному позиционированию, т.к. в этом случае используются данные не ближайших базовых станций, а всемирной сети станций, например, IGS (англ. International GNSS Service).
Корректирующая информация, созданная по данном такой сети, передается потребителю по спутниковому каналу связи или через Интернет, и позволяет ему определить свое местоположение с точностью около 1 см даже на удалении от базовых станций \cite{basile2018}. 
Недостатком метода является его долгая сходимость, обычно до получения первого точного навигационного решения требуется подождать около десяти минут.






