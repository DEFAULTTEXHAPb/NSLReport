\section*{Обозначения и сокращения}
\label{cha:acro}

\begin{longtable}{lll}
АГС  & - & астрономо-геодезическая сеть \\ 
АЗН-В& - & автоматическое зависимое наблюдение в режиме вещания \\ 
АКФ  & - & автокорреляционная функция \\ 
АРК  & - & автоматический радиокомпас \\ 
АРМ  & - & автоматизированное рабочее место \\ 
АЧХ  & - & амплитудно-частотная характеристика \\ 
БПРМ & - & ближняя приводная радиостанция с маяком \\ 
БПФС & - & блок приема и формирования сигналов \\ 
ВПП  & - & взлетно-посадочная полоса \\ 
ВПР  & - & высота принятия решения \\ 
ВРЛ  & - & вторичный радиолокатор \\ 
ВС   & - & воздушное судно \\ 
БЧХ  & - & Боуза-Чоудхури-Хоквингема (коды) \\ 
ГВПП & - & взлетно-посадочная полоса с грунтовым покрытием \\ 
ГНСС & - & глобальная навигационная спутниковая система \\ 
ГРМ  & - & глиссадный радиомаяк \\ 
ДИСС & - & доплеровский измеритель скорости и сноса \\ 
ДПРМ & - & дальняя приводная радиостанция с маяком \\ 
ЕСКД & - & единая система конструкторской документации \\ 
ИВПП & - & взлетно-посадочная полоса с искусственным покрытием \\ 
ИКАО & - & международная организация гражданской авиации \\ 
ИКД  & - & интерфейсный контрольный документ \\ 
ИНС  & - & инерциальные навигационные системы \\ 
ЗИП  & - & запасные части, инструменты и принадлежности \\ 
КГС  & - & курсо-глиссадная система \\ 
КГС  & - & космическая геодезическая сеть \\ 
КВС  & - & командир воздушного судна \\ 
КРМ  & - & курсовой радиомаяк \\ 
КСУ  & - & комплект средств управления \\ 
ЛА   & - & летательный аппарат \\ 
ЛРНС & - & локальная радионавигационная система \\ 
МВС  & - & минимальная высота снижения \\
МРМ  & - & маркерный радиомаяк \\
МСЭ  & - & международный союз электросвязи \\
НВО  & - & навигационно-временные определения \\
НИР  & - & научно-исследовательская работа \\
ОГ   & - & опорный генератор \\
ОКР  & - & опытно-конструкторская работа \\
ОНТД & - & отчетная научно-техническая документация \\
ОПРС & - & отдельная приводная радиостанция \\
ОРЛ  & - & обзорный радиолокатор \\
ПАВ  & - & поверхностная акустическая волна \\
ПЛ   & - & программируемая логика \\
ПЛИС & - & программируемая логическая интегральная схема \\
ПНК  & - & пилотажно-навигационный комплекс \\
ПП   & - & приемопередатчик \\
ПРЛ  & - & посадочный радиолокатор \\
ПРМГ & - & посадочная радиомаячная группа \\
ПРС  & - & приводная радиостанция \\
ПС   & - & псевдоспутник \\
ПС   & - & процессорная система \\
ПСП  & - & псевдослучайная последовательность \\
РЛЭ  & - & руководство по летной эксплуатации \\
РМ   & - & радиомаяк \\
РСБН & - & радиотехническая система ближней навигации \\
РСДН & - & радиотехническая система дальней навигации \\
РЭБ  & - & радио-электронная борьба \\
СДКМ & - & система дифференциальной коррекции и мониторинга \\
СнК  & - & система-на-кристалле \\
СП   & - & система посадки \\
СПМ  & - & спектральная плотность мощности \\
СПМО & - & специальное программно-математическое обеспечение \\
СПМО ПП & - & СПМО помехоустойчивого приема \\
СЧ   & - & стандарт частоты \\
СЧ   & - & составная часть \\
ТТЗ  & - & тактико-техническое задание \\
УМ   & - & усилитель мощности \\
УХЛ  & - & умеренный и холодный макроклиматический район \\
ФАГД & - & фундаментальная астрономо-геодезическая сеть \\
ФД   & - & функциональные дополнения (см. SBAS и GBAS) \\
BOC  & - & Binary Offset Carrier   \\
DMA  & - & Direct Memory Access   \\
DME  & - & Distance Measurement Equipment \\
EVS  & - & Enhanced Vision System \\
FAP  & - & Final Approach Point \\
FIS  & - & Flight Information Service \\
GBAS & - & Ground Based Augmentation System   \\
GLS  & - & GNSS Landing System   \\
ICAO & - & International Civil Aviation Organization \\ 
ILS  & - & Instrument Landing System (см. КГС метрового диапазона) \\
MLS  & - & Microwave Landing System (см. КГС сант-ого диапазона) \\
LDPC & - & Low-density parity-check code \\
NDB  & - & Non-directional beacon \\
PAPI & - & Precision Approach Path Indicator   \\
PL   & - & Programmable Logic   \\
PS   & - & Processor System   \\
SBAS & - & Satellite Based Augmentation System   \\
TIS  & - & Traffic Information Service  \\
TMBOC& - & Time Multiplexed Binary Offset Carrier   \\
WAM  & - & Wide Area Multilateration  \\
WGS  & - & World Geodetic System  \\
\end{longtable} 